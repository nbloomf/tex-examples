\documentclass{article}

\begin{document}

\LaTeX\ really shines when typesetting mathematics -- in fact it was designed, in part, because quality typeset mathematics before the 1970s was so expensive, while inexpensive personal tools like typewriters were so bad at two-dimensional notation. You can see this yourself: some books and papers from that era were even made using typewriters with math written in by hand. Badly typeset text is harder to understand and painful to read.

There are two different ways to include math in text. \emph{Inline} math is within a paragraph, and is delimited using dollar signs like so: $a + b = c$. \emph{Display} math is set by itself and centered on the page, and is delimited using backslash-square brackets, like so: \[ a + b = c. \]

Inline math is appropriate for short bits of notation that do not involve large symbols or complicated superscripts. If a bit of notation is too cramped or difficult to follow inline, use display style.

In math mode we have access to lots of symbols like Greek letters ($\alpha$, $\beta$, $\gamma$), operators ($\oplus$, $\otimes$, $\ast$), relations ($\cong$, $\sim$, $\equiv$) and more. We can also easily typeset superscripts ($x^2$, $2^{a+b}$) and subscripts ($x_1$). Most importantly \TeX\ has some powerful algorithms built in for deciding how expressions should be spaced, based on decades of experience in traditional printing -- so we don't have to think about it. Just type your math and move on.

\end{document}
