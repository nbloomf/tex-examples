\documentclass{article}

\begin{document}

Some information is best presented in a list. But not all lists are the same; different kinds of lists need different formatting. Plain \LaTeX\ provides three kinds of lists, depending on the kind of information they hold.

``Itemize'' is for sets of items with no particular order, like grocery lists. By default, the items in an itemize environment are printed with a bullet.
\begin{itemize}
\item This is a thing
\item Here is another thing
\item Want some more things? Have one!
\end{itemize}

``Enumerate'' is for ordered lists of items, like recipes or agendas. Take care to only use enumerate if the numeric labels make sense.
\begin{enumerate}
\item This is the first thing
\item This is the second thing
\item list all the things
\end{enumerate}

``Description'' is for lists of things that most naturally have textual labels, like glossaries. This is probably the least frequently used type of list.
\begin{description}
\item[foo] a meaningless term that stands for an arbitrary word
\item[bar] like ``foo'', also meaningless; sometimes called a metasyntactic variable
\item[baz] when you need a meaningless word, but foo and bar are already taken
\end{description}

\end{document}
